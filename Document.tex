\documentclass[letterpaper]{article}
\usepackage[latin1]{inputenc}
\usepackage{amsmath}
\usepackage{amsfonts}
\usepackage{amssymb}
\usepackage{graphicx}
\usepackage[left=2.00cm, right=2.00cm, top=2.00cm, bottom=2.00cm]{geometry}

\usepackage{Sweave}
\begin{document}

\input{Document-concordance}

\title{US-Mexico Border Apprehensions}
\author{Kevin Chan, Brianna Kincaid, Lauren Vanvalkenburg}
\date{}
\maketitle

Over the past 17 years, apprehensions at the US-Mexico border have reached a historic low. Data on U.S. Border Patrol Southwest Border Apprehensions released by the US Customs and Border Protrection (CBP) in 2017 revealed that apprehensions of individuals attempting to illegally cross the Southern border has decreased from 447,731 in 2010 to 303,196 in 2017. Some credit the decrease in apprehensions to President Donald Trump's enforcement of immigration laws. In April 2017, there were 11,129 apprehensions at the Southwest border, a 62\% drop from the previous April. February and March also saw a decrease in apprehensions, which has not been the case since 2000. 

\section{Data Import and Cleaning}

\subsection{Import Data}
\begin{Schunk}
\begin{Sinput}
> A2010 <- read.csv( "BP Apprehensions 2010.csv" , header = TRUE, stringsAsFactors = FALSE)
> A2017 <- read.csv("PB Apprehensions 2017.csv", header = TRUE, stringsAsFactors = FALSE)
\end{Sinput}
\end{Schunk}

\begin{table}[ht]
\centering
\scalebox{0.75}{
\begin{tabular}{p{0.7in}cccccccccccc}
  \hline
 & October & November & December & January & February & March & April & May & June & July & August & September \\ 
  \hline
Big Bend & 530.00 & 421.00 & 373.00 & 433.00 & 484.00 & 660.00 & 575.00 & 493.00 & 415.00 & 280.00 & 295.00 & 329.00 \\ 
  Del Rio & 1119.00 & 897.00 & 697.00 & 1234.00 & 1245.00 & 1874.00 & 1791.00 & 1718.00 & 1326.00 & 767.00 & 1095.00 & 931.00 \\ 
  El Centro & 2589.00 & 2412.00 & 2196.00 & 2688.00 & 2836.00 & 4408.00 & 3419.00 & 3126.00 & 2440.00 & 2331.00 & 2075.00 & 2042.00 \\ 
  El Paso & 1007.00 & 894.00 & 725.00 & 1124.00 & 1140.00 & 1528.00 & 1359.00 & 1380.00 & 1005.00 & 725.00 & 732.00 & 632.00 \\ 
  Laredo & 2613.00 & 2130.00 & 1802.00 & 2526.00 & 3173.00 & 4433.00 & 4528.00 & 3813.00 & 3475.00 & 1857.00 & 2819.00 & 2118.00 \\ 
  Rio Grande & 4236.00 & 3688.00 & 2987.00 & 3658.00 & 4845.00 & 7141.00 & 7139.00 & 7477.00 & 5595.00 & 3832.00 & 5329.00 & 3839.00 \\ 
  San Diego & 5017.00 & 4738.00 & 4636.00 & 6413.00 & 6982.00 & 9061.00 & 7115.00 & 5858.00 & 5092.00 & 5113.00 & 4528.00 & 4012.00 \\ 
  Tucson & 23197.00 & 16986.00 & 10907.00 & 16122.00 & 21266.00 & 31197.00 & 28579.00 & 22572.00 & 13160.00 & 10303.00 & 9280.00 & 8633.00 \\ 
  Yuma & 582.00 & 649.00 & 711.00 & 586.00 & 819.00 & 1059.00 & 732.00 & 608.00 & 447.00 & 401.00 & 262.00 & 260.00 \\ 
   \hline
\end{tabular}}
\caption{CBP Data for 2010} 
\end{table}

\begin{table}[ht]
\centering
\scalebox{0.75}{
\begin{tabular}{p{0.7in}cccccccccccc}
  \hline
 & October & November & December & January & February & March & April & May & June & July & August & September \\ 
  \hline
Big Bend & 697.00 & 603.00 & 477.00 & 473.00 & 383.00 & 357.00 & 413.00 & 552.00 & 378.00 & 492.00 & 563.00 & 614.00 \\ 
  Del Rio & 2106.00 & 1880.00 & 1817.00 & 1243.00 & 1104.00 & 746.00 & 589.00 & 740.00 & 761.00 & 760.00 & 798.00 & 932.00 \\ 
  El Centro & 2441.00 & 1850.00 & 1870.00 & 1796.00 & 1196.00 & 871.00 & 849.00 & 1134.00 & 1280.00 & 1478.00 & 1880.00 & 1988.00 \\ 
  El Paso & 3973.00 & 4105.00 & 3948.00 & 2779.00 & 1575.00 & 978.00 & 906.00 & 1032.00 & 1180.00 & 1395.00 & 1782.00 & 1540.00 \\ 
  Laredo & 3350.00 & 3194.00 & 2460.00 & 2265.00 & 1710.00 & 1256.00 & 1304.00 & 1722.00 & 1839.00 & 2120.00 & 2143.00 & 2097.00 \\ 
  Rio Grande & 22642.00 & 24686.00 & 23418.00 & 15580.00 & 7855.00 & 4147.00 & 3942.00 & 4882.00 & 5817.00 & 7107.00 & 8650.00 & 8836.00 \\ 
  San Diego & 2934.00 & 2947.00 & 3099.00 & 2927.00 & 1808.00 & 1356.00 & 1392.00 & 1724.00 & 1652.00 & 1764.00 & 2241.00 & 2242.00 \\ 
  Tucson & 5924.00 & 5912.00 & 4303.00 & 3357.00 & 2589.00 & 2148.00 & 1487.00 & 2199.00 & 2632.00 & 2177.00 & 2913.00 & 3016.00 \\ 
  Yuma & 2117.00 & 2034.00 & 1859.00 & 1156.00 & 534.00 & 336.00 & 245.00 & 534.00 & 548.00 & 894.00 & 1318.00 & 1272.00 \\ 
   \hline
\end{tabular}}
\caption{CBP Data for 2017} 
\end{table}

The CBP data we analyze here includes total apprehensions broken down by month and sector for both 2010 and 2017. The data includes statistics for Big Bend, Del Rio, El Centro, El Paso, Laredo, Rio Grande Valley, San Diego, Tucson, and Yuma. 

\subsection{Clean Data}
Here we are organizing the data with column and row totals with the appropriate row and column names.

\subsubsection{2010 Data}

Use the strings in column 1 as row names.
\begin{Schunk}
\begin{Sinput}
> rownames(A2010) <- A2010[,1]
\end{Sinput}
\end{Schunk}

Drop column 1
\begin{Schunk}
\begin{Sinput}
> A2010 <-  subset(A2010, select= -c(Sector))
\end{Sinput}
\end{Schunk}

Use rbind to add the sum across columns to the dataframe.
\begin{Schunk}
\begin{Sinput}
> A2010 <- rbind(A2010, colSums(A2010))
\end{Sinput}
\end{Schunk}

Drop that name that rbind assigns and rename the row that contains the column totals as "Total"
\begin{Schunk}
\begin{Sinput}
> rownames(A2010) <- c(rownames(A2010)[-length(rownames(A2010))], "Total")
\end{Sinput}
\end{Schunk}

Use cbind to add the sum across rows to the dataframe.
\begin{Schunk}
\begin{Sinput}
> A2010 <- cbind(A2010,rowSums(A2010))
\end{Sinput}
\end{Schunk}

Drop that name that cbind assigns and rename the column that contains the row totals as "Total"
\begin{Schunk}
\begin{Sinput}
> colnames(A2010) <- c(colnames(A2010)[-length(colnames(A2010))], "Total")
\end{Sinput}
\end{Schunk}

\subsubsection{2017 Data}

Use the strings in column 1 as row names.
\begin{Schunk}
\begin{Sinput}
> rownames(A2017) <- A2017[,1]
\end{Sinput}
\end{Schunk}

Drop column 1
\begin{Schunk}
\begin{Sinput}
> A2017 <-  subset(A2017, select= -c(Sector))
\end{Sinput}
\end{Schunk}

Use rbind to add the sum across columns to the dataframe.
\begin{Schunk}
\begin{Sinput}
> A2017 <- rbind(A2017, colSums(A2017))
\end{Sinput}
\end{Schunk}

Drop that name that rbind assigns and rename the row that contains the column totals as "Total"
\begin{Schunk}
\begin{Sinput}
> rownames(A2017) <- c(rownames(A2017)[-length(rownames(A2017))], "Total")
\end{Sinput}
\end{Schunk}

Use cbind to add the sum across rows to the dataframe.
\begin{Schunk}
\begin{Sinput}
> A2017 <- cbind(A2017,rowSums(A2017))
\end{Sinput}
\end{Schunk}

Drop that name that cbind assigns and rename the column that contains the row totals as "Total"
\begin{Schunk}
\begin{Sinput}
> colnames(A2017) <- c(colnames(A2017)[-length(colnames(A2017))], "Total")
\end{Sinput}
\end{Schunk}

\section{2010 and 2017 Compared}

\subsection{By Sector}

Here we compare the total apprehensions in each sector for 2010 and 2017. 
\medskip
Extracting sector data for 2010:
\begin{Schunk}
\begin{Sinput}
> year2010s <- t(as.data.frame(matrix(A2010[1:9,13])))
> colnames(year2010s) <- rownames(A2010[1:9,])
> colnames(year2010s)[6] <- "Rio Grande"
\end{Sinput}
\end{Schunk}
 
Extracting sector data for 2017:
\begin{Schunk}
\begin{Sinput}
> year2017s <- t(as.data.frame(matrix(A2017[1:9,13])))
> colnames(year2017s) <- rownames(A2017[1:9,])
> colnames(year2017s)[6] <- "Rio Grande"
\end{Sinput}
\end{Schunk}

Combining sector data for 2010 and 2017:
\begin{Schunk}
\begin{Sinput}
> year2010_17s <- rbind(year2010s, year2017s)
> row.names(year2010_17s) <- c("2010", "2017")
\end{Sinput}
\end{Schunk}

Creating a bar plot:
\begin{Schunk}
\begin{Sinput}
> barplot(as.matrix(year2010_17s), beside = TRUE, col = c("red", "blue"), bty="n",las=2)
> legend("topleft", c("2010","2017"), pch=15,  col=c("red","blue"),  bty="n")
> title("Border Patrol Apprehensions by Sector")
\end{Sinput}
\end{Schunk}
\includegraphics{Document-017}
    
The sector with the largest number of apprehensions for the year changed from Tucson in 2010 to Rio Grande Valley in 2017.

\subsection{By Month}
Here we compare the total apprehensions for each month in 2010 and 2017.

\medskip

Extracting monthly data for 2010:
\begin{Schunk}
\begin{Sinput}
> year2010m <- t(as.data.frame(matrix(unname(t(A2010)[1:12,10]))))
> colnames(year2010m) <- colnames(A2010[1:12])
\end{Sinput}
\end{Schunk}

Extracting monthly data for 2017:
\begin{Schunk}
\begin{Sinput}
> year2017m <- t(as.data.frame(matrix(unname(t(A2017)[1:12,10]))))
> colnames(year2017m) <- colnames(A2017[1:12])
\end{Sinput}
\end{Schunk}

Combining monthly data for 2010 and 2017:
\begin{Schunk}
\begin{Sinput}
> year2010_17m <- rbind(year2010m, year2017m)
> row.names(year2010_17m) <- c("2010", "2017")
\end{Sinput}
\end{Schunk}

Creating a bar plot:
\begin{Schunk}
\begin{Sinput}
> barplot(as.matrix(year2010_17m), beside = TRUE, col = c("red", "blue"), bty="n",las=2)
> legend("topleft", c("2010","2017"), pch=15,  col=c("red","blue"),  bty="n")
> title("Border Patrol Apprehensions by Month")
\end{Sinput}
\end{Schunk}
\includegraphics{Document-021}

As can be seen in the data, the trend throughout the year regarding total apprehensions across sectors changed drastically from 2010 to 2017. The two months with the highest apprehensions in 2010 (March and April) had the least amount of apprehensions in 2017. Furthermore, the month with the highest number of apprehensions in 2010 had more than 10,000 more apprehensions that the month ith the greatest umber of apprehensions in 2017.

\subsection{t-test}

\subsubsection{A comparison between the sector with most apprehensions for 2010 and the sector with most apprehensions in 2017}

First we must extract the total apprehensions by sector for 2010. 
\begin{Schunk}
\begin{Sinput}
> Sector_Totals_2010 <- A2010[1:9,13]
> names(Sector_Totals_2010) <- rownames(A2010[1:9,])
\end{Sinput}
\end{Schunk}

Then we do the same thing for 2017.
\begin{Schunk}
\begin{Sinput}
> Sector_Totals_2017 <- A2017[1:9,13]
> names(Sector_Totals_2017) <- rownames(A2017[1:9,])
\end{Sinput}
\end{Schunk}

Then we use these totals to determine the indeces of the sector with the greatest number of apprehensions. 
\begin{Schunk}
\begin{Sinput}
> MA_2017_index <- which(Sector_Totals_2017 == max(Sector_Totals_2017))
> MA_2010_index <- which(Sector_Totals_2010 == max(Sector_Totals_2010))
\end{Sinput}
\end{Schunk}

We use these indeces to extract the monthly data for the sectors with the greatest number of apprehensions in 2010 and 2017.
\begin{Schunk}
\begin{Sinput}
> MA_2017 <- A2017[MA_2017_index,1:12]
> MA_2010 <- A2010[MA_2010_index,1:12]
\end{Sinput}
\end{Schunk}

\noindent\rule{\textwidth}{0.4pt}
\medskip

We compared the sector with the most apprehensions for 2010 with the sector with most apprehensions in 2017 at a 5-percent significance level. The sectors with the highest apprehensions in 2010 and 2017 are Tuscan and Rio Grande Valley, respectively. 

\begin{Schunk}
\begin{Sinput}
> t.test(MA_2010,MA_2017)
\end{Sinput}
\end{Schunk}

\textbf{Welch Two Sample t-test}

data: MA2010 and MA2017\par
t = 1.9547\par
df = 21.973\par
p-value = 0.06346\par
alternative hypothesis: true difference in means is not equal to 0\par
95 percent confidence interval: -379.5935  12819.5935\par
\par
\bigskip
\textbf{sample estimates:}\par
mean of x: 17683.5\par
mean of y: 11463.5\par
\bigskip
The t test resulted in a p-value of .06346, which at 5-percent significance means the difference between the two sectors is not statistically significant.\par
\medskip

\noindent\rule{\textwidth}{0.4pt}
\medskip

We also decided to compare 2010 with 2017 Tuscan and 2010 with 2017 Rio Grande Valley.\par

\begin{Schunk}
\begin{Sinput}
> t.test(MA_2010,A2017[MA_2010_index,1:12])
\end{Sinput}
\end{Schunk}

\textbf{Welch Two Sample t-test}

data: Tuscan 2010 and Tuscon 2017\par
t = 6.4303\par
df = 11.781\par
p-value = 3.545e-05\par
alternative hypothesis: true difference in means is not equal to 0\par
95 percent confidence interval: 9551.716  19372.450\par
\par
\bigskip
\textbf{sample estimates:}\par
mean of x: 17683.500\par
mean of y: 3221.417\par
\bigskip

The t test between 2010 and 2017 Tuscan resulted in a p-value of .00003545, which is statistically significant. Which means there is a considerable difference between the amount of apprehensions in Tuscan from 2010 to 2017. 

\begin{Schunk}
\begin{Sinput}
> t.test(MA_2017,A2010[MA_2017_index,1:12])
\end{Sinput}
\end{Schunk}

\textbf{Welch Two Sample t-test}

data: Rio Grande Valley 2010 and Rio Grande Valley 2017\par
t = 2.7789\par
df = 11.846\par
p-value = 0.01686\par
alternative hypothesis: true difference in means is not equal to 0\par
95 percent confidence interval: 1392.65  11573.35\par
\par
\bigskip
\textbf{sample estimates:}\par
mean of x: 11463.5\par
mean of y: 4980.5 \par
\bigskip
The t test between 2010 and 2017 Rio Grande Valley resulted in a p-value of .01686, which is also statistically significant. There is a noticeable difference between the number of apprehensions in Rio Grande Valley from 2010 to 2017.

\subsubsection{A comparison between the 3 month periods with the most apprehensions in 2010 and 2017}

\begin{Schunk}
\begin{Sinput}
> col <- c("Oct-Dec", "Jan-Mar", "Apr-Jun","Jul-Sep")
> Monthly_Totals_2010 <- (t(A2010)[1:12,10])
> A2010_3 <- rbind(sum(Monthly_Totals_2010[1:3]),sum(Monthly_Totals_2010[4:6]),sum(Monthly_Totals_2010[7:9]),sum(Monthly_Totals_2010[10:12]))
> Monthly_Totals_2017 <- (t(A2017)[1:12,10])
> A2017_3 <- rbind(sum(Monthly_Totals_2017[1:3]),sum(Monthly_Totals_2017[4:6]),sum(Monthly_Totals_2017[7:9]),sum(Monthly_Totals_2017[10:12]))
\end{Sinput}
\end{Schunk}

In 2010, the three month period with the most apprehensions is January, February and March. In 2017, the three month period with the most apprehensions is October, November and December. We ran a t test comparing these three month periods with a 5-percent significance level.

\begin{Schunk}
\begin{Sinput}
> t.test(Monthly_Totals_2010[4:6],Monthly_Totals_2017[1:3])
\end{Sinput}
\end{Schunk}

\textbf{Welch Two Sample t-test}

data: Jan/Feb/March 2010 and Oct/Nov/Dec 2017\par
t = 0.095848\par
df = 2.0908\par
p-value = 0.932\par
alternative hypothesis: true difference in means is not equal to 0\par
95 percent confidence interval: -32101.2  33627.2\par
\par
\bigskip
\textbf{sample estimates:}\par
mean of x: 46311.67\par
mean of y: 45548.67\par
\bigskip
The t test resulted in a p-value of .932, which is not statistically significant. There is not a significant difference in the amount of apprehensions in January, February and March of 2010 and October, November and December of 2017.

\section{Overall Trends}

\subsection{Data Cleaning}

Here we import monthly summaries of apprehensions across all sectors for years 2000 to 2017. 

\begin{Schunk}
\begin{Sinput}
> A2000.2017 <- read.csv("PB monthly summaries.csv", header = TRUE, stringsAsFactors = FALSE)
\end{Sinput}
\end{Schunk}

Use the strings in column 1 as row names.
\begin{Schunk}
\begin{Sinput}
> rownames(A2000.2017) <- A2000.2017[,1]
\end{Sinput}
\end{Schunk}

Drop column 1
\begin{Schunk}
\begin{Sinput}
> A2000.2017 <-  subset(A2000.2017, select= -c(year))
\end{Sinput}
\end{Schunk}

Reorder the rows so that the years list from 2000 to 2017.
\begin{Schunk}
\begin{Sinput}
> A2000.2017 <- A2000.2017[18:1,]
\end{Sinput}
\end{Schunk}

\begin{table}[ht]
\centering
\scalebox{0.80}{
\begin{tabular}{p{0.25in}cccccccccccc}
  \hline
 & October & November & December & January & February & March & April & May & June & July & August & September \\ 
  \hline
2000 & 91410 & 76196 & 71252 & 185979 & 211328 & 220063 & 180050 & 166296 & 115093 & 113956 & 114312 & 97744 \\ 
  2001 & 82632 & 67709 & 55081 & 125090 & 152229 & 170580 & 142813 & 122927 & 89131 & 83602 & 84648 & 59276 \\ 
  2002 & 37812 & 32506 & 31501 & 79793 & 95724 & 126992 & 121921 & 97424 & 78655 & 76661 & 82557 & 68263 \\ 
  2003 & 61792 & 47731 & 37824 & 86925 & 96869 & 98399 & 75359 & 88690 & 75530 & 79284 & 84486 & 72176 \\ 
  2004 & 65391 & 57894 & 43614 & 92521 & 110669 & 154981 & 135468 & 118726 & 94590 & 92165 & 93246 & 80017 \\ 
  2005 & 75913 & 65135 & 48406 & 93020 & 113775 & 143048 & 140062 & 115823 & 90786 & 94954 & 96733 & 93741 \\ 
  2006 & 83557 & 70975 & 52673 & 101195 & 125046 & 160696 & 126538 & 105450 & 68366 & 59641 & 59751 & 58084 \\ 
  2007 & 60713 & 51594 & 40527 & 71934 & 79268 & 114137 & 104465 & 88504 & 71338 & 66782 & 59795 & 49581 \\ 
  2008 & 51339 & 42209 & 31802 & 59028 & 73483 & 89770 & 91566 & 69233 & 53854 & 49472 & 48541 & 44708 \\ 
  2009 & 42938 & 32780 & 25947 & 44502 & 49211 & 67342 & 58493 & 50884 & 46044 & 43843 & 43522 & 35359 \\ 
  2010 & 40890 & 32815 & 25034 & 34784 & 42790 & 61361 & 55237 & 47045 & 32955 & 25609 & 26415 & 22796 \\ 
  2011 & 26165 & 22405 & 19429 & 23926 & 28786 & 42014 & 36251 & 31236 & 27166 & 23170 & 24166 & 22863 \\ 
  2012 & 25612 & 23368 & 18983 & 25714 & 31579 & 42218 & 40628 & 36966 & 30669 & 26978 & 27567 & 26591 \\ 
  2013 & 28929 & 27636 & 23243 & 26921 & 35042 & 47293 & 48212 & 43856 & 34436 & 33230 & 33797 & 31802 \\ 
  2014 & 35312 & 31896 & 29528 & 28668 & 36403 & 49596 & 51502 & 60683 & 57862 & 40708 & 31388 & 25825 \\ 
  2015 & 26450 & 24641 & 25019 & 21514 & 24376 & 29791 & 29750 & 31576 & 29303 & 28388 & 30239 & 30286 \\ 
  2016 & 32724 & 32838 & 37014 & 23758 & 26072 & 33316 & 38089 & 40337 & 34450 & 33723 & 37048 & 39501 \\ 
  2017 & 46184 & 47211 & 43251 & 31576 & 18754 & 12195 & 11127 & 14519 & 16087 & 18187 & 22288 & 22537 \\ 
   \hline
\end{tabular}}
\caption{Total apprehensions for all sectors for years 2000 to 2017} 
\end{table}

Remove the rownames and column names from the data frame in order to prepare to make it a time series.
\begin{Schunk}
\begin{Sinput}
> rownames(A2000.2017) <- c()
> A2000.2017 <- unname(A2000.2017)
\end{Sinput}
\end{Schunk}

\subsection{Time Series}

Here we use the monthly summaries data in order to create a time series.
\begin{Schunk}
\begin{Sinput}
> ts2 <- as.vector(t(A2000.2017))
> time_series <- ts(ts2, start = c(2000,10), frequency=12)
> ts.plot(time_series, gpars=list(xlab="year", ylab="Apprehensions", lty=c(1:3)))
> meanbyyear <- rowMeans(A2000.2017)
> years <- c(2000:2017)
> lines(years,meanbyyear,col="red")
> title("Border Patrol Apprehensions Year")
> legend("topright", c("Average Apprehensions"), pch=15,  col="red",  bty="n")
\end{Sinput}
\end{Schunk}
\includegraphics{Document-036}

\end{document}
